% !Mode:: "TeX:UTF-8"
%%%%%%%%%%%%%%%%%%%%%%%%%%%%%%%
%%       Visão geral     %%
%%%%%%%%%%%%%%%%%%%%%%%%%%%%%%%
\section{Visão Geral}

O sistema de emissão de certificados digitais pessoa ICPEdu (Infraestrutura de Chaves Públicas para Ensino e Pesquisa) foi desenvolvido para gerenciar o ciclo de vida do certificado pessoa, desde de sua emissão à sua revogação e a manutenção da respectiva Lista de Certificados Revogados (LCR).

O ciclo de vida do certificado digital tem início no momento da sua emissão. 
A revogação de um certificado é iniciada quando as informações contidas no certificado não são mais válidas e/ou necessitam ser atualizadas. Dentre os motivos da revogação podemos citar:

\begin{itemize}
  \item Alteração de dados que identificam o proprietário do certificado;
  \item Desvinculação do proprietário do certificado;
  \item Desvinculação do proprietário do certificado com a entidade que o emitiu;
  \item Atualização dos algoritmos criptográficos;
\end{itemize}

O final do processo de revogação do certificado digital é sua publicação na Lista de Certificados Revogados (LCR).

Os usuários estão divididos em 3 perfis:

\begin{itemize}
  \item \textbf{Administrador}: responsável por criar os usuários, cadastrar módulos de segurança criptográficos, entre outras operações;
  \item \textbf{Operador}: responsável pela operação da organização para a qual foi cadastrado.
  \item \textbf{Usuário final}: responsável por solicitar e emitir seu certificado digital pessoal e solicitar a revogação do mesmo.
\end{itemize}

Por questões de segurança, foi observada a necessidade da criação de 3 usuários com privilégios distintos, separando o controle operacional do administrativo. Este Plano de Teste de Software tem o propósito de definir as atividades e procedimentos necessários para a validação dos requisitos definidos para o projeto ICPEdu. Além disso, faz-se necessário também a descrição dos testes do ponto de vista do usuário final do serviço, que corresponde ao terceiro e último perfil.

O presente plano de testes pode ser aplicado em qualquer um dos ambientes implantados (Teste, Homologação e Produção). O resultado dos testes deve ser reportado para o gerente que solicitou a execução do plano de testes.


\nocite{*}